% -------------------------------
% Latex report template.
% -------------------------------

\documentclass[10pt, titlepage, oneside, a4paper]{article}
\usepackage[T1]{fontenc}
\usepackage[swedish]{babel}
\usepackage[utf8]{inputenc}
\usepackage{listings}
\usepackage{fancyhdr}
\usepackage{float}
\usepackage{courier}
\usepackage{csquotes}
\usepackage{pgffor}
\usepackage{etoolbox}
\usepackage{biblatex}

\usepackage{hyperref} % import hyperref last

% -------------------------------
% Editable settings section.
% Example values follow the fields. Remove define to remove field in doc.

\def\school{SCHOOL} % "UMEÅ UNIVERSITY
\def\inst{INSTITUTION} % "Department of Computing Science"
\def\typeofdoc{TYPE} % "Report"
% \def\bibs{refs.bib} % Comma seperated list of biblatex resources.

\def\course{COURSENAME} % "Operating Systems"
\def\courselen{LENGTH} % "7.5 HP"
\def\courseid{COURSEID} % "5DV176"

\def\pretitle{PRETITLE} % "Assignment 1"
\def\doctitle{TITLE} % Syscall
\def\reportdate{\today} % "March 25, 2021"
% \def\version{VERSION} % "1.0"

\def\names{NAME1,NAME2} % Comma separated list of students
\def\emails{EMAIL1@DOMAIN.COM,EMAIL2@DOMAIN.COM} % Student emails, same order as names.
% \def\footernames{FNAME1, FNAME2} % Used in footer if defined, defaults to \names if not def.

\def\teacher{TEACHER NAME} % Course responsible
\def\assistants{TA1,TA2} % Comma separated list of TAs.

% End editable settings section.
% -------------------------------


% Make all links black.
\hypersetup{colorlinks=true,allcolors=black}

% Set up lstlisting styles.
\lstset{basicstyle=\footnotesize\ttfamily,breaklines=true,frame=single}

% Count names and TAs.
\ifdefined\names
    \newcounter{names}
    \foreach \name in \names {\stepcounter{names}}
\fi
\ifdefined\assistants
    \newcounter{tas}
    \foreach \ta in \assistants {\stepcounter{tas}}
\fi

% Define footernames if not defined.
\ifx\footernames\undefined
    \def\footernames{\names}
\fi

% Add bib resources.
\ifdefined\bibs
    \foreach \bib in \bibs {\addbibresource{\bib}}
\fi

\begin{document}

    \begin{titlepage}
        \thispagestyle{empty}
        \begin{center}
            \begin{large}
                \begin{tabular}{@{}p{\textwidth}@{}}
                    \textbf{\ifdefined\school \school\fi \hfill \ifdefined\reportdate \reportdate\fi} \\
                    \textbf{\ifdefined\inst \inst\fi \hfill \ifdefined\version \version\fi} \\
                    \ifdefined\typeofdoc \textbf{\typeofdoc} \\\fi
                \end{tabular}
            \end{large}
        \end{center}
        \vspace{2em}
        \begin{center}
            \ifdefined\courseid
                \ifdefined\pretitle
                    \LARGE{\courseid \ - \pretitle} \\
                \else
                    \LARGE{\courseid} \\
                \fi
            \else
                \ifdefined\pretitle
                    \LARGE{\pretitle} \\
                \fi
            \fi
            \ifdefined\course
                \huge{\textbf{\course \ifdefined\courselen , \courselen \fi}} \\
            \fi
            \ifdefined\doctitle
                \vspace{2em}
                \LARGE{\doctitle} \\
            \fi
            \vspace{3em}
            \begin{large}
                \ifdefined\names
                    \ifnum\value{names}=1%
                        \begin{tabular}{ll}
                            \textbf{Namn} & \names \\
                            \\
                            \ifdefined\emails
                                \textbf{Epost} & \texttt{\emails} \\
                                \\
                            \fi
                        \end{tabular}
                    \else % More than one student.
                        \def\nametbl{}
                        \foreach \name [count=\nameidx] in \names {                        
                            \xappto\nametbl{& \name & }
                            \ifdefined\emails
                                \foreach \email [count=\emailidx] in \emails {
                                    \ifnum\nameidx=\emailidx{}%
                                        \xappto\nametbl{\email}
                                    \fi
                                }
                            \fi
                            \xappto\nametbl{ \\ \\}
                        }
                        \begin{tabular}{lll}
                            \textbf{Namn} \nametbl
                        \end{tabular}
                    \fi
                \fi
            \end{large}

            \vfill

            \ifdefined\teacher
                \large{\textbf{Kursansvarig}} \\
                \mbox{\large{\teacher}} \\
            \fi

            \ifdefined\assistants
                \vspace{2em}
                \large{\textbf{Handledare}} \\
                \foreach \grader in \assistants
                {
                    \mbox{\large{\grader}} \\
                }
            \fi
        \end{center}
    \end{titlepage}

    \ifdefined\footernames \lfoot{\footnotesize{\footernames}}\fi

    \ifdefined\reportdate \rfoot{\footnotesize{\reportdate}}\fi
    \ifdefined\doctitle \lhead{\sc\footnotesize\doctitle}\fi
    \rhead{\nouppercase{\sc\footnotesize\leftmark}}
    
    \pagestyle{fancy}
    \renewcommand{\headrulewidth}{0.2pt}
    \renewcommand{\footrulewidth}{0.2pt}
    
    % Cover sheet
    % Abstract
    % Content
    % Problem Specification
    % Usage and user's guide
    % Algorithm Description
    % Systems Description
    % Limitations on the Solution
    % Problems and Reflections
    % Test runs
    % References
    % Source Code
    % Attachments

    % \begin{figure}[H] %usepackage{float}
    %     \centering
    %     \fboxsep=0mm %padding thickness
    %     \fboxrule=2pt %border thickness
    %     \includegraphics[width=0.8\textwidth]{fig.png}
    %     \caption{Figure caption.}
    %     \label{fig:}
    % \end{figure}

    % \begin{lstlisting}[language=C]
    % \end{lstlisting}

    \pagenumbering{roman}
    \tableofcontents
 
    \setlength{\parindent}{0pt}
    \setlength{\parskip}{10pt}

    \newpage
    \pagenumbering{arabic}

    \normalsize

    \section{Introduktion}


\end{document}